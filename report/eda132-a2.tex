%%%%%%%%%%%%%%%%%%%%%%%%%%%%%%%%%%%%%%%%%
% Journal Article
% LaTeX Template
% Version 1.3 (9/9/13)
%
% This template has been downloaded from:
% http://www.LaTeXTemplates.com
%
% Original author:
% Frits Wenneker (http://www.howtotex.com)
%
% License:
% CC BY-NC-SA 3.0 (http://creativecommons.org/licenses/by-nc-sa/3.0/)
%
%%%%%%%%%%%%%%%%%%%%%%%%%%%%%%%%%%%%%%%%%

%----------------------------------------------------------------------------------------
%	PACKAGES AND OTHER DOCUMENT CONFIGURATIONS
%----------------------------------------------------------------------------------------

\documentclass[twoside]{article}

\usepackage{lipsum} % Package to generate dummy text throughout this template

\usepackage[sc]{mathpazo} % Use the Palatino font
\usepackage[T1]{fontenc} % Use 8-bit encoding that has 256 glyphs
\linespread{1.05} % Line spacing - Palatino needs more space between lines
\usepackage{microtype} % Slightly tweak font spacing for aesthetics

\usepackage[hmarginratio=1:1,top=32mm,columnsep=20pt]{geometry} % Document margins
\usepackage{multicol} % Used for the two-column layout of the document
\usepackage[hang, small,labelfont=bf,up,textfont=it,up]{caption} % Custom captions under/above floats in tables or figures
\usepackage{booktabs} % Horizontal rules in tables
\usepackage{float} % Required for tables and figures in the multi-column environment - they need to be placed in specific locations with the [H] (e.g. \begin{table}[H])
\usepackage{hyperref} % For hyperlinks in the PDF

\usepackage{lettrine} % The lettrine is the first enlarged letter at the beginning of the text
\usepackage{paralist} % Used for the compactitem environment which makes bullet points with less space between them

\usepackage{abstract} % Allows abstract customization
\renewcommand{\abstractnamefont}{\normalfont\bfseries} % Set the "Abstract" text to bold
\renewcommand{\abstracttextfont}{\normalfont\small\itshape} % Set the abstract itself to small italic text

\usepackage{titlesec} % Allows customization of titles
\renewcommand\thesection{\Roman{section}} % Roman numerals for the sections
\renewcommand\thesubsection{\Roman{subsection}} % Roman numerals for subsections
\titleformat{\section}[block]{\large\scshape\centering}{\thesection.}{1em}{} % Change the look of the section titles
\titleformat{\subsection}[block]{\large}{\thesubsection.}{1em}{} % Change the look of the section titles

\usepackage{fancyhdr} % Headers and footers
\pagestyle{fancy} % All pages have headers and footers
\fancyhead{} % Blank out the default header
\fancyfoot{} % Blank out the default footer
\fancyhead[C]{EDA132 $\bullet$ Lund Institute of Technology $\bullet$ February 2015} % Custom header text
\fancyfoot[RO,LE]{\thepage} % Custom footer text

%----------------------------------------------------------------------------------------
%	TITLE SECTION
%----------------------------------------------------------------------------------------

\title{\vspace{-15mm}\fontsize{24pt}{10pt}\selectfont\textbf{Assignment 2: Probabilistic Reasoning}} % Article title

\author{
\large
\textsc{
Magnus T\"{o}rnquist \&
Anton Karlstedt}\\[2mm] % Your name
\vspace{-5mm}
}
\date{}

%----------------------------------------------------------------------------------------

\begin{document}

\maketitle % Insert title

\thispagestyle{fancy} % All pages have headers and footers

%\newpage
\section{System} % OLD TEXT BELLOW

This project is written in java and is divied into four files: Locate.java, TestDataGeneration.java, TestLocalisation.java and World.java. 

World.java is used to emulate the robots behavior with methods for sensing the location of the robot, moving the robot and for testing purposes get the actual location of the robot. 
The TestDataGeneration.java file is mainly used for testing the functionality of World.java.

Locate.java is where the main part of the assignment is, it contains the method for localising. TestLocalistion.java test this.

\section{Discussion}
The book, page 593[1] shows that for a matrix of size 42x42 the estimation error is generally close to 1 when you use more then 40 observations. In our case this has been mostly true with the exception 
of a some extreme outliers.

\section{Where you can find it}
The code can be found in /h/d9/s/dt08ak0/kurser/eda132/A2, with the source files in src/main and the jar in build/. The jar file emulates and tries to locate a robot in an 40x40 matrix 
after 1000 steps. The jar file also accepts three arguments: the dimensions of the matrix (2 arguments) and the amount of steps the robot should take before it tries to locate it.

%----------------------------------------------------------------------------------------
%	REFERENCE LIST
%----------------------------------------------------------------------------------------

\begin{thebibliography}{99} % Bibliography - this is intentionally
                            % simple in this template
                            
\bibitem{book}  Artificial Intelligence: A Modern Approach, 3/e, by Stuart Russell and Peter Norvig, ISBN-10: 0132071487.

\end{thebibliography}

%-------------------------------------------------------

\end{document}
